\documentclass[12pt, a4paper]{article}

\usepackage{helvet}
\usepackage[margin=1.0in]{geometry}

\usepackage{lstautogobble}

\usepackage{soul}

\renewcommand{\familydefault}{\sfdefault}

\setul{}{1pt}
\setlength{\parskip}{1em}
\setlength{\parindent}{0pt}

\begin{document}

  {\LARGE\bfseries Planbot}

  \vspace{-0.5cm}
  \rule{\linewidth}{2pt}

  {\large\bfseries White paper - April 2017}

  {\large\bfseries Authors: Adam Staveley, Simon Tucker}

  Planbot is a chatbot developed by a team consisting of two University of
  Reading graduates. It aims to provide users of different levels of experience
  with easily accessible property information and documents, with a current
  focus on planning. This white paper outlines how and why Planbot was
  developed, while also presenting an image of how its future may look.

    \parskip 4em

  {\large\bfseries\ul{The motivation behind a property-based chatbot}}

    \parskip 1em

  Planbot came about through a mutual passion for both technology and property.
  On the one hand, there exists impending wide-scale automation through rapid
  development in technologies such as machine learning and blockchain, and on
  the other, a property industry with a planning system largely playing catch
  up. The first thought, and indeed the first iteration of Planbot, was a
  natural language chatbot that could infer meaning from user sentances; a
  chatbot that had the ability to converse with its users about property and
  inform them of the planning system. The problem with this quickly became
  an issue of complexity.

  By and large, the user requires accurate information and most likely wishes to
  consume that information as quickly as possible whilst minimising frustration.
  The choice became apparent during the development process that a shift in
  thinking was required. The following sections will draw upon this shift
  further.

  With that said, the two `hands' outlined above still remain firmly in mind.
  Nobody can be sure of the impact of automation on job losses: each week a new
  report is released detailing potential losses of 40\%, 60\%, 80\%. Nor is the
  future of artificial intelligence and impending singularity set in stone,
  just as distributed ledger technlogy needs to overcome various difficulties.
  These technologies, however, should be embraced. The ability to use advanced
  statistics to predict actions based on large datasets lends itself well to
  chatbots. The immutable, open and distributed data that blockchain technology
  provides potential access to, allows for the use of far greater amounts of
  data. Planbot, therefore, moves forward with these technologies in mind.

  Planbot aims to appeal to a wide range of users: planners, developers and
  self-builders. The chatbot aims to provide these users with a tool for quickly
  finding information about definitions and planning mechanisms while also,
  (perhaps the main draw) being able to request various documents pertinent to
  planning. This is believed to be a solid foundation from which Planbot can
  grow in the future. Our approach to Planbot was one of transparency: ease of
  access to open information. Planbot therefore acts as a front-end user
  interface to an open source of data, which is also accessible via a
  public-facing API.

    \parskip 3em

  {\large\bfseries\ul{What is a Chatbot?}}

    \parskip 1em

  A chatbot is an interactive computer program designed to mimic (to various
  extents) human conversation. They are often used to provide 24/7 customer
  service, though general purpose chatbots have been developed in the past
  and will continue to be in the future. They most often take form on instant
  messaging platforms and as such require quick responses to user queries,
  usually with a character limit for each message. Chatbots can either be
  rules-based where the program listens for triggers and uses them to run an
  appropriate function on the 'back-end', or else they can use Natural Language
  Processing (NLP) to learn from conversations as more recent experiments like
  Microsoft's 'Tay' and 'Zo' have shown (with differing levels of success).

    \parskip 3em

  {\large\bfseries\ul{Current functions of Planbot}}

    \parskip 1em

  Planbot releases with three core functions which aim to entice professionals
  into beginning to understand and conceptually apply automated processes to
  roles within their fields. The three functions are:

    \begin{itemize}
      \item {\bfseries Definitions}: The user can ask for definitions to a
        wide range of key terms often found in planning and property.
        Currently there are over 750 definitions a user can ask for,
        sourced from the Planning Portal and the Lexicon of PRS, BtR and
        Property by Richard Berridge.
      \item {\bfseries Information}: Planbot can find a user information on
        two planning mechanisms: use classes and permitted development. A
        user might ask for further details on the C3 use class or whether
        planning permission is needed for extensions, for example. The
        information here is also sourced from Planning Portal.
      \item {\bfseries Documents}: Three types of documents can be asked
        for:
          \begin{itemize}
            \item {\bfseries Policy \& Legislation}: Statutory documents,
              such as the Planning Act 2008, or national/regional policy,
              such as the National Planning Policy Framework, can be
              asked for by the user.
            \item {\bfseries Local Plans}: By inputting a local planning
              authority, or alternatively a postcode, Planbot will find
              the local plan (or extant local development framework) for that
              area and send a link for the relevant PDF or webpage to the user.
          \end{itemize}
      \end{itemize}

    \parskip 5em

  {\large\bfseries\ul{How Planbot works}}

    \parskip 1em

  Planbot has been built using the Python programming language. Python was used
  for a variety of reasons: its syntax makes it faster to write than a number
  of other languages and, in turn, hopefully easier to understand for those
  coming into the project. The language also has a wide range of 3rd party
  libraries available which extend its default capabilities. Some of the
  libraries used in the project include spaCy, a NLP framework, and Bottle, a
  lightweight web application framework. This means that the majority of the
  Planbot infrastructure can be built upon Python with relative ease. Python
  has been used to build a range of other websites and web apps, including for
  instance, social media sites Reddit and Instagram.

  The NLP module, spaCy, allows Planbot to parse meaning from the user's query
  more successfully. Early in testing, it was discovered that though there
  exist, for instance, a myriad of possible definitions a user can ask for,
  unless the user knows with certainty the exact string or substring that
  points to that definition, the bot will not return it. Broadly speaking, an
  algorithm comprised of sentiment analysis, least distance matching and
  substring matching is used to maximise the ability of the bot to answer a
  user's query, or at the very least present the user with a range of options
  to select from.

  Planbot did use the wit.ai API early on, which uses NLP to parse context and
  intents from user queries. This allows, for instance, the user to say 'Can
  you find me the definition for green infrastructure', which will be broken
  down into an intent of 'find me a definition' and a context of 'green
  infrastructure'. The problem with this becomes apparent when 1) trying to
  communicate what can and can't be achieved through the chatbot, 2)
  increasing its complexity by adding further lines of enquirity to the chatbot
  and, 3) predicting responses between similar questions which should garner
  different outcomes. Simply put, to ask a question and be given a completely
  disparate response was never the goal with Planbot.

  As a result of the above, there was a shift in design philosophy to a more
  structured approach, where the user is given quick reply options and
  responses are centred around one or two word answers, to communicate the
  bot's capabilities more transparently. In line with Facebook Messenger's bot
  guidelines, menus were incorporated which make it much easier to manage
  context through conversation branches on our end. Having developed a
  deterministic, but (hopefully) scalable bot engine for selecting repsonses
  to user queries, we have far greater license over the chatbot than using a
  bot framework with NLP logic stored on a different server.

  Last but certainly not least, it was decided that Planbot would be open
  source early on in the project. With the source code being free to modify,
  the potential for collaborators to join the project increases, while the
  license chosen, the GNU General Public Licence version 3, dictates that any
  resulting projects should also be open source. With one of the goals of the
  project being increased transparency, we believe that the choice of GPLv3
  meets this criteria.

    \parskip 3em

  {\large\bfseries\ul{Current limitations and future developments}}

    \parskip 1em

  There are several elements of the project that can and will be improved upon
  in the future. One key point to make is that of managing expectations. This
  is not solely concerned with our chatbot, but with chatbots more generally.
  It is possible that Microsoft's 'Tay' and 'Zo' have warped public perception
  about what the majority of chatbots are currently capable of. The mimicry of
  human conversation and the 'learning from' aspect of those bots do capture
  the imagination, but the reality is that this can lead to user frustration in
  the short term for a chatbot that centres around relaying much-needed
  information. That is why, as previously detailed, a more structured
  approach was taken. The aim is not to have our chatbot hold a user's hand
  throughout the conversation, but to minimise frustration for all users,
  even if this betrays the 'chat' in chatbot to a certain extent. It is
  possible, in the future, that Planbot shifts back to a more 'fluid'
  conversation structure, but for now this style of chatbot is preferred.

  The current feature set of Planbot aims to showcase a range of uses
  available to different users. At this stage, however, it is primarily focused
  on the self-builder, with features that aim to educate rather than to serve.
  Planbot is an `open book' of sorts: being an experimental community project
  its content could change dramatically over the next year and beyond. further
  features could include, for instance, the ability to search within PDF
  documents for more specific information. As well as this, there is evidently
  more planning information to be included in Planbot. In an earlier version
  of the bot, a feature that allowed users to fetch recent market reports from
  sources such as Savills and Cushman Wakefield was included, however it was
  decided that this was at odds with core Planbot features. As such this
  feature is likely to form its own entity in the future.

  Looking out over the horizon, we aim to establish Planbot as a viable
  platform for retrieving property-related information, preferred over using a
  search engine. We also wish to establish and incorporate our own services,
  through the use of machine learning.

  This white paper has hopefully described and explained the process we have
  gone through in developing Planbot. Given the rapidly changing field,
  Planbot is a project that has been designed with flexibility in mind,
  with the ability to be shaped by many stakeholders. This makes its
  development a challenge, but one that we relish. We hope that Planbot
  can assert itself as a viable alternative to current means of finding
  property information and believe that its current aggregation of
  information sets it apart in that regard. In the future, we see Planbot
  as being a powerful agent within the industry through the wider use of data.


\end{document}
